\documentclass[12pt, a4paper, titlepage]{extarticle}
\usepackage{styles/style}
\usepackage{styles/symbols}
\usepackage{styles/titlepage}

\linespread{1.05}
\begin{document}

	\drawtitle
	\pagestyle{empty}

	\vspace*{-10pt}

	\begin{center}

		\textbf{РЕЦЕНЗИЯ}

		\vspace*{-4pt}
		на выпускную квалификационную работу

		% \vspace*{-4pt}
		\underlinefillbox{<<Реализация сервиса, предоставляющего услуги по защите интеллектуальной}
		\underlinefillbox{собственности посредством обфускации ПО>>}

		\vspace*{4pt}
		выполненную обучающимся 

		\vspace*{-4pt}
		Дальневосточного федерального университета

		\vspace{5pt}
		\underlinefillbox{Акуленко Валерием Васильевичем}

	\end{center}

	\textbf{\textit{Актуальность работы}}

		Работа посвящена созданию сервиса для защиты интеллектуальной собственности приложений с помощью обфускации программного обеспечения, что актуально в условиях возрастающих угроз кибербезопасности и необходимости защиты программных решений от несанкционированного использования и реверс-инжиниринга.

	\textbf{\textit{Характеристика работы}}

		Работа включает анализ предметной области и существующих решений, формирование требований к обфускатору и веб-сервису, а также реализацию Web-сервиса программного продукта. Основное внимание уделено разработке сервиса: реализованы серверная часть на C++, включающая в себя сам обфускатор, а также клиентская часть на основе классических Web-технологий для управления уровнем обфускации. Значительное внимание уделено алгоритмам обфускации, включая неявное изменение констант, изменение потока выполнения и преобразование математических операций. Работа имеет теоретическую основу, описанную в главе 1, и практическую реализацию, подробно изложенную в главе 3.

	\textbf{\textit{Замечания}}

		При чтении выпускной квалификационной работы возникли следующие замечания:
		\begin{enumerate}
			\item В разделе тестирования производительности сервиса, стоило бы рассмотреть нагрузочное тестирование в условиях высоких нагрузок для оценки масштабируемости и устойчивости системы.

			\item В описании пользовательского интерфейса, рекомендуется уделить больше внимания удобству использования (usability) и оценке пользовательского опыта (UX).
		\end{enumerate}

	\pagebreak
	\textbf{\textit{Заключение}}

		Дипломная работа Акуленко Валерия Васильевича на тему <<Реализация сервиса, предоставляющего услуги по защите интеллектуальной собственности посредством обфускации ПО>> соответствует требованиям, предъявляемым к выпускным квалификационным работам, и заслуживает оценки <<отлично>>. Автору рекомендуется продолжить разработку проекта и рассмотреть возможность дальнейшего обучения в магистратуре.

	\textbf{Рецензент}

	\underlinesimple{(должность, уч. звание)}{\text{\hside{10pt}{ассистент}}}
	\hfill \usersignature{\hside{10pt}{Бузалов П.К.}}

	\vspace{10pt}
	\textbf{Дата:} 
	\vspace{5pt}
	\docdatemonthsp{\hspace{90pt}}{e}

	\vspace{35pt}
	\hspace{10cm}{\fontsize{9}{9} \selectfont М.П.}

\end{document}