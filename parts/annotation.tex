В данной работе исследуется двухпараметрическая задача мультипликативного управления для нелинейного уравнения реакции--диффузии--конвекции. На его основе которого строится численный алгоритм решения стационарных задач для диффузионных моделей распространения вирусов. Как известно, в 2020м году весь мир накрыла пандемия вируса COVID-19, и вопрос математического моделирования физических систем, позволяющих предсказать дальнейшие события на их основе является наиболее актуальным. Для конкретных коэффициента реакции в уравнении и коэффициента массобмена в граничном условии получена система оптимальности. На основе анализа данной системы выводятся оценки локальной устойчивости оптимальных решений относительно малых возмущений как функционала качества, так и одной из заданных функций, не являющихся управлением. Затем, на основе полученных результатов можно сформировать сходимость численных алгоритмов для решения прямых и обратных задач.

\pagebreak