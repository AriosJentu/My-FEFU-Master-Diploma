В процессе реализации работы была рассмотрена двухпараметрическая задача мультипликативного управления для нелинейного уравнения реакции--диффузии--конвекции. Для конкретных коэффициента реакции в уравнении и коэффициента массобмена в граничном условии получена система оптимальности. На основе анализа данной системы выводятся оценки локальной устойчивости оптимальных решений относительно малых возмущений как функционала качества, так и одной из заданных функций, не являющихся управлением. Затем, на основе полученных результатов была сформирована сходимость численных алгоритмов для решения прямых и обратных задач. Также был проведен численный эксперимент на основе модели распространения вирусов, и сделаны соотвествующие выводы по поводу результатов эксперимента.
% Отметим, что интерес к свойству bang--bang вызван исследованием задач управления, в которых из практических соображений не используется регуляризация. В частности, такая постановка задач управления используется при исследовании прикладных задач тепловой и электромагнитной маскировки (см., например, \cite{lit21} и ссылки там). 