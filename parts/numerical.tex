\subsection{Постановка задачи}

	Рассмотрим следующую систему нелинейных уравнений с частными производными, рассматриваемую в ограниченной области $\Omega \subset \R^2$:
	\begin{equation}
		\label{numeqs}
		\begin{cases}
			\displaystyle \dpart{S}{t} &= - \kappa S I - D S \Delta I, \\ 
			\displaystyle \dpart{I}{t} &= \kappa S I + D S \Delta I - \dfrac{1}{\tau} I, \\
			\displaystyle \dpart{R}{t} &= \dfrac{1}{\tau} I,
		\end{cases} ~ \text{в} ~ \Omega \times (0, T], ~ T > 0.
	\end{equation}
	со следующими начальными и краевыми условиями Дирихле:
	\begin{equation}
		\begin{cases} 
			S\ont_{t_0} &= S_0(\mathbf{x}), ~ S\ont_{\Gamma} = S_{\Gamma}(t, \mathbf{x}), \\
			I\ont_{t_0} &= I_0(\mathbf{x}), ~ I\ont_{\Gamma} = I_{\Gamma}(t, \mathbf{x}), \\
			R\ont_{t_0} &= R_0(\mathbf{x}), ~ R\ont_{\Gamma} = R_{\Gamma}(t, \mathbf{x}).
		\end{cases}
	\end{equation}
	Здесь $\Gamma$ -- граница области $\Omega$. Функция $S$ имеет смысл вероятности восприимчивости к заражению особей в пространстве, $I$ -- соответственно вероятность заражения, и $R$ -- вероятность восстановленния после заболевания. $\kappa$ -- скорость передачи инфекции, $\tau$ -- характеристическое время восстановления. $D$ -- коэффициент диффузии. Данная система была предложена в \cite{lit23}, и получена из системы обыкновенных диффференциальных уравнений, используя пространственно усредненную концентрацию инфицированных вокруг рассматриваемой точки вместо локальной в коэффициентах реакции. Вероятность заражения особи может изменяться только в том случае, если особь восприимчива, и зависит от числа ее инфицированных ближайших соседей.

	Данная система удовлетворяет закону сохранения численности особей во времени:
	\[ \dpart{S}{t} + \dpart{I}{t} + \dpart{R}{t} = 0. \]

	В общем случае, система \ref{numeqs} представляет из себя нелинейную систему нестационарных уравнений диффузии--реакции, для которых справедлив анализ, проведенный ранее.

	В работах \cite{lit22,lit23} были исследованы вопросы сходимости численных алгоритмом решения таких задач, и проведен анализ устойчивости решений. В качестве численного алгоритма в статье была проведена симуляция Монте-Карло. В данной работе сведем данную задачу к стационарной, и на основе рассмотреной ранее теории построим численный алгоритм решения нелинейной стационарной краевой задачи для уравнения (2) системы (\ref{numeqs}) на основе метода конечных элементов.  

\subsection{Реализация}

	Сведем систему (\ref{numeqs}) к системе стационарных уравнений. Для этого введем итерационный процесс. Положим $S_k$, $I_k$ и $R_k$ -- состояниями в момент времени $t_k$. Тогда, фиксируя значение $h = \Delta t = t_{k+1} - t_k$ -- приращения времени, получим следующую систему стационарных дифференциальных уравнений:
	\begin{equation}
		\label{numeqs2}
		\begin{cases}
			\displaystyle S_k &= S_{k-1} + h \cdot \pares{- \kappa S_k I_k - D S_k \Delta I_k}, \\ 
			\displaystyle I_k &= I_{k-1} + h \cdot \pares{\kappa S_k I_k + D S_k \Delta I_k - \dfrac{1}{\tau} I_k}, \\
			\displaystyle R_k &= R_{k-1} + h \cdot \dfrac{1}{\tau} I_k,
		\end{cases} ~ \text{в} ~ \Omega,
	\end{equation}
	со следующими начальными условиями для итерационного процесса, и краевыми условиями Дирихле для каждого уравнения этой системы:
	\begin{equation}
		\begin{cases} 
			S_0 &= S_0(\mathbf{x}), ~ S_k\ont_{\Gamma} = S_{\Gamma}(t_k, \mathbf{x}) = S_k(\mathbf{x})_{\Gamma}, \\
			I_0 &= I_0(\mathbf{x}), ~ I_k\ont_{\Gamma} = I_{\Gamma}(t_k, \mathbf{x}) = I_k(\mathbf{x})_{\Gamma}, \\
			R_0 &= R_0(\mathbf{x}), ~ R_k\ont_{\Gamma} = R_{\Gamma}(t_k, \mathbf{x}) = R_k(\mathbf{x})_{\Gamma}.
		\end{cases}
	\end{equation}

	Можно заметить, что первое и третье уравнения системы (\ref{numeqs2}) на самом деле являются обыкновенными рекуррентными уравнениями, но при этом зависят от решения второго уравнения. Второе же уравнение представляет собой линейное уравнения реакции--диффузии, зависимое от решения в предыдущий момент времени. Таким образом была получена линейная задача, которую решим с помощью метода конечных элементов.

	В качестве рассматриваемой области $\Omega$, для простоты, возьмем квадрат $1 \times 1$ безразмерных единиц. В качестве зависимости параметров скорости передачи инфекции и характеристического времени восстановления, выберем, согласно \cite{lit23}, $\kappa \tau = 5$, $D = 1$. Полагая вероятности $S$, $I$ и $R$ безразмерными, положим следующий закон сохранения:
	\[ S + I + R = 1. \]

	Начальное распределение в области и краевые условия выберем следующими:
	\begin{equation}
		\label{numbounds}
		\begin{cases} 
			S_0 &= 1-I_0 , ~ S_k\ont_{\Gamma} = \max{\pares{1-\dfrac{t_k}{2T}, 0}}, \\
			I_0 &= 0.1 e^{-5\pares{\abs{\mathbf{x}}-0.5}^2}, ~ I_k\ont_{\Gamma} = 0, \\
			R_0 &= 0, ~ R_k\ont_{\Gamma} = 0.
		\end{cases}
	\end{equation}

	С физической точки зрения, такие условия удовлетворяют следующей модели: Концентрация основной части зараженных особей в начальный момент времени находится в центре области $\Omega$, и плавно распределена в округе. Все остальные особи в это время являются подверженными. В области в начальный момент времени отсутствуют восстановленные после заболевания особи. На границе происходит следующее: подверженные особи все меньше приходят в зону заражения со временем, и к половине рассматриваемого периода все возможные особи полностью покинут границы области. На границе отсутствуют инфицированные и восстановленные особи, что говорит об изолированности системы относительно зараженных.

	Положим $h = 0.05$, $T = 1$. Рассматривать будем поведение инфицированных особей в пространстве. Построим слабую формулировку для уравнения (2) задачи (\ref{numeqs2}) с краевыми условиями (\ref{numbounds}). Для этого, домножим на вспомогательную функцию $v \in H^1(\Omega)$, и проинтегрируем по частям с помощью формул Грина:
	\begin{equation}
		\label{numweakform}
		\begin{split}
			&\pares{ \vphantom{\frac{1}{\tau}} D h S_k \grad{I_k}, \grad{v} } + \pares{ \pares{1 + \frac{1}{\tau} - h \kappa S_k} \cdot I_k, v} = \pares{\vphantom{\frac{1}{\tau}} I_{k-1}, v}, \\
			& I_k \ont_{\Gamma} = 0, ~ \forall{v} \in H^1(\Omega).
		\end{split}
	\end{equation}
	Здесь
	\[ \pares{u, v} = \int_{\Omega} u \cdot v ~ d\mathbf{x}. \]

	Реализацию проведем на системе компьютерной математики \textit{FreeFEM++}. Графики промежуточных результатов расположены в разделе <<Приложение>>.

	Анализируя численно полученное решение, сделаем вывод, что алгоритм дает результаты, схожие с теми, которые были получены другими методами в \cite{lit22,lit23}. Так же, согласно теории, изложенной ранее, выполняется принцип минимума и максимума. Основываясь на заданных параметрах, относительно полученного решения можно сделать вывод, что концентрация инфицированных особей в пространстве уменьшается.