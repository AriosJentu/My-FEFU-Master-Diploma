В данной работе исследуется двухпараметрическая задача мультипликативного управления для нелинейного уравнения 
реакции--диффузии--конвекции, рассматриваемого в ограниченной области $\Omega \subset \R^3$: 
\begin{equation}
	\label{4a}
		- \div{(\lambda(\mathbf{x}) \grad{\vp})} 
		+ \mathbf{u} \cdot \grad{\vp}
		+ k (\vp, \mathbf{x}) \vp 
		= f \text{ в } \Omega, 
\end{equation}
Предполагается, что  граница $\Gamma$ области $\Omega$ состоит из двух частей $\Gamma_D$ и $\Gamma_N$ и уравнение (\ref{4a}) рассматривается при смешанных краевых условиях:
\begin{equation}
	\label{4}
		\vp = \psi \text{ на } \Gamma_D, ~
		\lambda (\mathbf{x}) 
		\left( \partial \vp / \partial n + \alpha(\vp, \mathbf{x})\vp \right)
		=\chi \text{ на } \Gamma_N.
\end{equation}
Здесь функция $\vp$ имеет смысл концентрации загрязняющего вещества, $\mathbf{u}$ -- заданный вектор скорости,  $f$ -- объемная плотность внешних источников вещества, $\lambda (\mathbf{x})$ -- коэффициент диффузии, $k(\vp, \mathbf{x})$ -- коэффициент реакции, $\alpha(\vp, \mathbf{x})$ -- коэффициент массобмена, функция $\chi$ имеет смысл плотности граничных источников. Ниже на задачу (\ref{4a}), (\ref{4}) при заданных функциях $\lambda, k, f, \alpha, \chi$ и $\psi$ будем ссылаться как на задачу 1.
%${\bf x} \in \overline \Omega$,

В \cite{lit15} доказана глобальная разрешимость задачи 1 при условии, что нелинейности, порождаемые зависимыми от решения коэффициентами, являются монотонными. При этом условии имеет место нелокальная единственность решения задачи 1, а также удается установить принцип максимума и минимума для концентрации $\vp$.

Для задачи 1 в \cite{lit15} была сформулирована задача мультипликативного управления для коэффициента реакции, имеющего вид произведения: $k(\vp, \mathbf{x}) = \beta(\mathbf{x}) k_0 (\vp)$. Функция $\beta(\mathbf{x})$ играла роль одного из мультипликативных управлений. Другим управлением был коэффициент диффузии $\lambda (\textbf{x})$. Разрешимость задачи управления, как и разрешимость краевой задачи, доказана для коэффициентов реакции и массообмена общего вида, имеющих операторную конструкцию (см. ниже). В случае, когда коэффициенты, зависящие от решения, а также функционалы качества дифференцируемы по Фреше, для экстремальных задач выводятся системы оптимальности. На основе анализа одной из полученных систем для оптимальных решений конкретной (однопараметрической) задачи управления установлено свойство релейности или
справедливость принципа bang-bang (см. о смысле этого термина ниже или в \cite{lit7,lit8,lit15}.

Настоящая работа дополняет статью \cite{lit15} исследованием устойчивости (единственности) оптимальных решений рассматриваемой задачи управления. Для конкретных коэффициентов $k(\vp,\cdot)$ и $\alpha(\vp)$ при использовании регуляризации минимизируемых фунционалов выводятся оценки локальной устойчивости оптимальных решений относительно малых возмущений как конкретных функционалов качества, так и заданной функции $f$. При отсутствии указанных возмущений из полученных оценок вытекает локальная единственность оптимального решения.

Стоит отметить, что свойство релейности оптимального управления установлено в \cite{lit15} для экстремальной задачи без использования регуляризации и без условия малости ее исходных данных.

При анализе рассматриваемых задач будем использовать функциональные пространства Соболева $H^s(D)$, $s \in \R$. Здесь $D$ обозначает область $\Omega$, либо некоторую подобласть $Q \subset \Omega$, либо часть $\Gamma_D$ границы $\Gamma$.

Через $\norm{\cdot}_{s,Q}, \abs{\cdot}_{s,Q}$ и $\pares{\cdot, \cdot}_{s,Q}$ будем обозначать норму, полунорму и скалярное произведение в $H^s (Q)$. Нормы и скалярные произведения в $L^2 (Q)$, $L^2 (\Omega)$ либо в $L^2 (\Gamma_N)$ будем обозначать соответственно через $\norm{\cdot}_Q$ и $\pares{\cdot, \cdot}_Q$, $\norm{\cdot}_\Omega$ и $\pares{\cdot, \cdot}$ либо $\norm{\cdot}_{\Gamma_N}$ и $\pares{\cdot, \cdot}_{\Gamma_N}$. 
Пусть 
\[
	L_{+}^{p} (D) = \braces{
		k \in L^p (D): k \ge 0 
	}, ~ p \ge 3/2, 
\] 
\[ 
	Z = \braces{ 
		\mathbf{v} \in L^4 (\Omega)^3: 
		\div{\mathbf{v}} = 0 \text{ в } \Omega, 
		~ \mathbf{v \cdot n}\ont_{\Gamma_N} = 0 
	},
\] 
\[ 
	H^{s}_{\lambda_0} (\Omega) = \braces{ 
		\lambda \in H^{s} (\Omega): \lambda \ge \lambda_0 > 0 \text{ в } \Omega 
	}, s > 3/2, 
\] 
\[ 
	\cT= \braces{ 
		\vp \in H^1 (\Omega): \vp \ont_{\Gamma_D} = 0 
	}.
\]

Предположим, что выполняются следующие условия:
\begin{enumerate}[(i)]

	\item \label{rconds1}
		$\Omega$ -- ограниченная область в  $\R^3$ с границей $\Gamma \in C^{0,1}$, состоящей из замыканий двух непересекающихся открытых участков $\Gamma_D$ и $\Gamma_N$ ($\Gamma = \overline \Gamma_D \cup \overline \Gamma_N$, $\Gamma_D \cap \Gamma_N = \emptyset$), при этом поверхностная мера $\meas{\Gamma_D} >0$ и граница $\partial \Gamma_D$ участка $\Gamma_D$ состоит из конечного числа липшицевых кривых или является $n$--угольником;

	\item \label{rconds2}
		$\lambda \in H^{s}_{\lambda_0} (\Omega)$, $s> 3/2$, $\mathbf{u} \in Z$, $f \in L^2(\Omega)$, $\psi \in H^{1/2} (\Gamma_D)$, $\chi \in L^2 (\Gamma_N)$.

	\item \label{rconds3}
		Для любой функции $v \in H^1(\Omega)$ справедливо вложение $k \pares{v, \cdot} \in L^{p}_{+} (\Omega)$ для некоторого $p \ge 3/2$, не зависящего от $v$, и на любом шаре $B_r = \braces{
			v \in H^1 (\Omega): \norm{v}_{1,\Omega} \le r 
		}$ радиуса $r$ выполняется неравенство
		\[
			\norm{k \pares{v_1, \cdot} - k \pares{v_2, \cdot}}_{L^p (\Omega)} 
			\le L_1 \norm{v_1 - v_2}_{L^4(\Omega)} \quad
			\forall v_1, v_2 \in B_r.
		\]
		Здесь константа $L_1$ зависит от $r$, но не зависит от $v_1, v_2 \in B_r$.

	\item \label{rconds4}
		Для любой функции $w \in H^1(\Omega)$ справедливо вложение $\alpha \pares{w, \cdot} \in L^{q}_{+} (\Gamma_N)$ для некоторого $q \ge 2$, не зависящего от $w$, и на любом шаре $S_a = \braces{ w \in H^1 (\Omega): \norm{w}_{1,\Omega} \le a }$ радиуса $a$ справедливо неравенство
		\[
			\norm{ \alpha \pares{w_1, \cdot} - \alpha \pares{w_2, \cdot} }_{L^q (\Gamma_N)} \le
			L_2 \norm{ w_1 - w_2 }_{L^2(\Gamma_N)} \quad \forall w_1, w_2 \in S_a.
		\]
		Здесь константа $L_2$ зависит от $a$, но не зависит от $w_1, w_2 \in S_a$.

\end{enumerate}

Будем предполагать, что нелинейности $k \pares{\vp,\cdot} \vp$ и $\alpha \pares{\vp, \cdot} \vp$ являются монотонными в следующем смысле:
\begin{enumerate}[(i)]
	\setcounter{enumi}{4}

	\item \label{rconds5}
		$\pares{ k (\vp_1, \cdot) \vp_1 - k (\vp_2, \cdot) \vp_2, \vp_1 - \vp_2} \ge 0$ для всех $\vp_1, \vp_2 \in H^1 (\Omega)$;

	\item \label{rconds6}
		$ \pares{ \alpha (\vp_1, \cdot) \varphi_1 - \alpha (\vp_2, \cdot) \vp_2, \vp_1 - \vp_2}_{\Gamma_N} \ge 0$ для всех $\vp_1, \vp_2 \in H^1 (\Omega)$.

\end{enumerate}

Пусть так же функции $ k (\vp, \cdot) $ и $\alpha (\vp, \cdot)$ ограничены в том смысле, что существуют положительные константы $A_1, B_1$, зависящие от $k$, и $A_2, B_2$, зависящие от $\alpha$, такие, что
\begin{enumerate}[(i)]
	\setcounter{enumi}{6}

	\item \label{rconds7}
		$\norm{ k (\vp, \cdot) }_{L^p (\Omega)} \le A_1 \norm{\vp}_{1,\Omega}^r + B_1$ для всех $\vp \in H^1 (\Omega)$ при $p \ge 3/2$, $r \ge 0$ ;

	\item \label{rconds8}
		$\norm{\alpha (\vp, \cdot)}_{L^q (\Gamma_N)} \le A_2 \norm{\vp}_{1,\Omega}^l + B_2$ для всех $\vp \in H^1 (\Omega)$ при $q \ge 2$, $l \ge 0$.

\end{enumerate}

Отметим, что условия (\ref{rconds3}), (\ref{rconds5}) и (\ref{rconds7}) описывают оператор, действующий из $H^1 (\Omega)$ в $L^p(\Omega)$, $p \ge 3/2$, позволяющий учитывать достаточно произвольную зависимость коэффициента реакции, как от концентрации $\vp$, так и от пространственной переменной $\mathbf{x}$. Например,
\[
	\tk_1 = \vp^2 ~ 
	(\text{или } \tk_1 = \vp^2 \abs{\vp}) 
	\text{ в подобласти } Q \subset \Omega 
	\text{ и } \tk_1 = k_0 (\mathbf{x}) \in L^{3/2}_+ (\Omega \setminus \overline Q) 
	\text{ в } \Omega \setminus \overline Q.
\]

В свою очередь, условия (\ref{rconds4}), (\ref{rconds6}) и (\ref{rconds8}) задают оператор, действующий из $H^1 (\Omega)$ в $L^q (\Gamma_N)$, $q \ge 2$, который позволяет учитывать зависимость коэффициента $\alpha$ от $\vp$ и $\mathbf{x}$. Например,
\[
	\ta_1 = \abs{\vp} ~ \text{ на } \Gamma_0 \subset \Gamma_N
	\text{ и } \ta_1 = \alpha_0 (\mathbf{x}) 
	\in L^2_+ (\Gamma_N \setminus \overline \Gamma_0) 
	\text{ в } \Gamma_N \setminus \overline \Gamma_0.
\]

Напомним также, что в силу теоремы вложения Соболева пространство $H^1 (\Omega)$ вкладывается в пространство $L^s (\Omega)$ непрерывно при $s \le 6$ и компактно при $s < 6$ и с некоторой константой $C_s$, зависящей от $s$ и $\Omega$, справедлива оценка
\begin{equation}
	\label{1.4arn}
	\norm{\vp}_{L^s (\Omega)} \le C_s \norm{\vp}_{1,\Omega} \quad 
	\forall \vp \in H^1 (\Omega).
\end{equation}

Пространство $H^{1/2} (\Gamma_N)$ вкладывается в пространство $L^q (\Gamma_N)$ непрерывно при $q \le 4$ и компактно при $q < 4$. В силу непрерывности оператора следа $\gamma: H^1 (\Omega) \to H^{1/2} (\Gamma_N)$ (и его сужения $\gamma\ont_{\Gamma_N}$ на $\Gamma_N \subset \Gamma$) с константой $\tilde C_q$, зависящей от $q$ и $\Gamma_N$, справедлива оценка
\begin{equation}
	\label{1.5arn}
	\norm{\vp}_{L^q (\Gamma_N)} \le \tilde C_q \norm{\vp}_{1,\Omega} \quad
	\forall \vp \in H^1 (\Omega).
\end{equation}

Наконец, поскольку при $ r > 1 $ пространство $ H^r (\Gamma_N) $ непрерывно и компактно вкладывается в $L^p (\Gamma_N)$, где $p \le \infty$, то из непрерывности оператора частичного следа $\gamma \ont_{\Gamma_N} : H^{r+1/2} (\Omega) \to H^{r} (\Gamma_N)$ вытекает оценка:
\begin{equation}
	\label{5RRR}
	\norm{\lambda}_{L^p (\Gamma_N)} \le \hat C \norm{\lambda}_{s,\Omega} \quad 
	\forall \lambda \in H^s (\Omega), \quad  s = r + 1/2 > 3/2.
\end{equation}

Справедлива следующая техническая лемма (см. \cite{lit15}).

\lemma{
	\label{lemm11}
	При выполнении условий (\ref{rconds1}), (\ref{rconds2}), $\mathbf{u} \in Z$, $\lambda \in H^{s}_{\lambda_0} (\Omega)$, $s> 3/2$, $k_1 \in L^{p}_{+} (\Omega)$, $p \ge 3/2$, $\alpha_1 \in L^q_+ (\Gamma_N)$, $q \ge 2$, существуют положительные константы $C_0$, $\delta_0$, $\gamma_1$, $\gamma_p$, зависящие от $\Omega$ или от $\Omega$ и $p$ или от $\Omega, \Gamma_N$ и $q$, с которыми справедливы соотношения
	\begin{equation}
		\label{s1}
		\begin{split}
			& \abs{\pares{\lambda \grad{\vp}, \grad{\eta}}} \le C_0 \norm{\lambda}_{s,\Omega}
			\norm{\vp}_{1,\Omega} \norm{\eta}_{1,\Omega}, \\ 
			& \abs{\pares{\mathbf{u} \cdot \grad{\vp}, \eta}} \le 
			\gamma_1 \norm{\mathbf{u}}_{L^4 (\Omega)^3} \norm{\vp}_{1,\Omega} \norm{\eta}_{1,\Omega}, \\
			& \abs{\pares{k_1 \vp, \eta}} \le \gamma_p \norm{k_1}_{L^p(\Omega)} \norm{\vp}_{1,\Omega} \norm{\eta}_{1,\Omega},
		\end{split}
	\end{equation}
	\begin{equation}
		\label{s2}
		\abs{\pares{\lambda \alpha_1 \vp, \eta}} \le 
		\gamma_q \norm{\lambda}_{s,\Omega}
		\norm{\alpha_1}_{L^q (\Gamma_N)} 
		\norm{\vp}_{1,\Omega} \norm{\eta}_{1,\Omega} ~ 
		\forall \vp, \eta \in H^1 (\Omega),
	\end{equation}
	\begin{equation}
		\label{s22}
		\abs{\pares{\chi, h}} \le \gamma_2 \norm{\chi}_{\Gamma_N} \norm{h}_{1,\Omega} ~ \forall \chi \in L^2 (\Gamma_N), ~ h \in H^1 (\Omega),
	\end{equation}
	\begin{equation}
		\label{s3}
		\pares{\mathbf{u} \cdot \grad{\vp}, \vp} = 0 ~ 
		\pares{\lambda \grad{\vp}, \grad{\vp}} \ge \lambda_* \norm{\vp}_{1,\Omega}^2 ~ 
		\forall \vp \in {\cT}, ~ 
		\lambda_* \equiv \delta \lambda_0.
	\end{equation}
}

Умножим уравнение (\ref{4a}) на $h \in \cT$ и проинтегрируем по $\Omega$,
применяя формулу Грина. Учитывая (\ref{4}), получим
\begin{equation}
	\label{2.4a}
	\begin{split}
		\pares{\lambda \grad{\vp}, \grad{h}} + 
		\pares{k (\vp, \cdot) \vp, h} + 
		\pares{\mathbf{u} \cdot \grad{\vp}, h} + 
		\pares{\lambda \alpha (\vp, \cdot) \vp, h}_{\Gamma_N} &= \pares{f,h} + \pares{\chi, h}_{\Gamma_N}
		\\ 
		&\forall h \in {\cT}, ~ \vp\ont_{\Gamma_D} = \psi.
	\end{split}
\end{equation} 

\definition{
	Функцию $\vp \in H^1 (\Omega)$, удовлетворяющую (\ref{2.4a}), назовем слабым решением задачи 1.
}

Для доказательства разрешимости задачи 1 в \cite{lit15} использовалась следующая лемма.

\lemma{
	\label{lemm12}
	Пусть выполняются условия (\ref{rconds1}). Тогда для любой функции $\psi \in H^{1/2} (\Gamma_D)$ существует функция $\vp_0 \in H^1 (\Omega)$ такая, что $\vp_0 = \psi$ на $\Gamma_D$ и с некоторой константой $C_\Gamma$, зависящей от $\Omega$ и $\Gamma_D$, справедлива оценка $\norm{\varphi_0}_{1,\Omega} \le C_\Gamma \norm{\psi}_{1/2,\Gamma_D}$.
}

Справедлива следующая теорема (см. \cite{lit15}).

\theorem{
	\label{thref11}
	При выполнении условий (\ref{rconds1})--(\ref{rconds8}) существует единственное слабое решение $\vp \in H^1 (\Omega)$ задачи 1, для которого справедлива оценка
	\begin{equation}
		\label{est_phi}
		\norm{\vp}_{1,\Omega} \le M_\vp \equiv 
		C_* M_l + C_\Gamma \norm{\psi}_{1/2,\Gamma_D},
	\end{equation}
	где $C_\Gamma$ -- константа из леммы \ref{lemm12} и
	\begin{equation}
		\label{l_nepr}
		\begin{split}
			M_l &\equiv
			C_2 \norm{f}_\Omega + \gamma_2 \norm{\chi}_{\Gamma_N} +
			(
				C_0 C_{\Gamma} 
				\norm{\lambda}_{s,\Omega} +
				\gamma_1 C_\Gamma 
				\norm{\mathbf{u}}_{L^4 (\Omega)^3} 
			) 
			\norm{\psi}_{1/2,\Gamma_D} + \\
			& + C_\Gamma 
			(
				\gamma_p 
				(
					A_1 C_\Gamma^r \norm{\psi}_{1/2,\Gamma_D}^r + B_1
				)
				+ \gamma_q \norm{\lambda}_{s,\Omega}
				(
					A_2 C_\Gamma^l \norm{\psi}_{1/2,\Gamma_D}^l + B_2
				) 
			) \norm{\psi}_{1/2,\Gamma_D},
		\end{split}
	\end{equation}
}