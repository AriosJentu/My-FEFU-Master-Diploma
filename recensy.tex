\documentclass[12pt, a4paper, titlepage]{extarticle}
\usepackage{styles/style}
\usepackage{styles/symbols}
\usepackage{styles/titlepage}

\linespread{1.05}
\begin{document}

	\drawtitle
	\thispagestyle{empty}

	\vspace*{-10pt}

	\begin{center}

		\textbf{РЕЦЕНЗИЯ}

		\vspace*{-4pt}
		на выпускную квалификационную работу

		\vspace*{-4pt}
		\underline{<<Численный анализ диффузионных моделей распространения вирусов>>}

		\vspace*{4pt}
		выполненную обучающимся 

		\vspace*{-4pt}
		Дальневосточного федерального университета

		% \vspace*{-4pt}
		\underline{Максимовым Павлом Александровичем}

	\end{center}

	\textbf{\textit{Актуальность работы}}

		Работа посвящена теоретичекому и численному исследованию краевых и экстремальных задач для нелинейных моделей реакции-диффузии-конвекции. При этом представлены приложения такого типа моделей к исследования распространения вирусов. Так что акутальность работы не вызывает сомнений.
		% Работа Максимова П.А. посвящена исследованию двухпараметрической задачи мультипликативного управления для нелинейного уравнения реакции-диффузии-конвекции, на основе которого строится численный алгоритм решения стационарных задач для диффузионных моделей распространения вирусов. Как известно, в 2020м году весь мир накрыла пандемия вируса COVID-19, и вопрос математического моделирования физических систем, позволяющих предсказать дальнейшие события на их основе является наиболее актуальным. 

		% В работе Максимова П.А. выводится система оптимальности для конкретных коэффициента реакции в уравнении и коэффициента массобмена в граничном условии. На основе анализа данной системы выводятся оценки локальной устойчивости оптимальных решений относительно малых возмущений как функционала качества, так и одной из заданных функций, не являющихся управлением.

	\textbf{\textit{Характеристика работы}}

		Представленная работа действительно содержит результаты по теоретическому исследованию краевых и экстремальных задач для нелинейного уравнения реакции-диффузии-конвекции. Часть результатов опубликовна в ЖВМ в 2021 г. Результаты, связанные с выводом оценок локальной устойчивости оптимальных решений подготовлены к публикованию. Также опубликован иттерационный алгоритм решения задачи управления, построенный на основе системы оптимальности (Scopus). Представлены результаты вычислительных экспериментов по моделям, описывающии распространение вирусов.

		% Работа Максимова П.А. состоит из аннотации, введения, пяти разделов, заключения, списка литературы. Объем текста составляет 36 страниц.

		% Во введении формируется актуальность исследований, проводимых в данной работе.

		% В первом разделе рассматривается постановка краевой задачи, для которой будут проводиться дальнейшие исследования. Так же здесь исследуется разрешимость поставленой задачи.

		% Во втором разделе ставится мультипликативная задача управления, и проводится исследование экстремальной задачи. В результате выводится принцип минимума для решений краевой задачи.

		% В третьем разделе выводится основное свойство оптимальности для решений задачи управления. Здесь выводятся оценки для решения и управлений для экстремальной задачи.

		% В четвертом разделе выводятся оценки локальной устойчивости оптимальных решений. Доказывается устойчивость решений, на основе которых можно проводить численные алгоритмы.

		% В пятом разделе приводится пример исследуемых моделей. Проводится численное решение стационарной задачи.

		% Используемые Максимовым П.А. численные алгоритмы основаны на методах конечных элементов, и являются стандартом для исследования линейных диффузионных задач. С помощью простых итерационных процессов, дополняющих этот метод достигается численное решение нелинейных задач.

		% В заключении формируются основные результаты квалификационной работы.

		% Данная работа производит положительное впечатление. Автор демонстрирует эрудированность в широких областях исследуемой проблематики. Максимов П.А. выполнил научное исследование в соответствии с поставленными задачами. 

	\textbf{\textit{Замечания}}

		При чтении диссертации возникли следующие замечания:
		\begin{enumerate}
			\item Автору следовало бы лучше структуировать работу. Такое пожелание является следствием недостаточной практики в структурировании. Со временем, я думаю, ситуация будет становиться лучше.
			
			\item На мой взгляд, несколько преувеличен прикладной смысл моделей распростанения вирусов на примере модели применяемой для исследования поведения вируса <<COVID-19>>, даже для пика их актуальности.
		\end{enumerate}

	\textbf{\textit{Заключение}}

		Выпускная квалификационная работа Максимова П.А. по теме <<Численный анализ диффузионных моделей распространения вирусов>> соответствует требованиям, предъявляемым к выпускным квалификационным работам, и заслуживает оценки <<\textit{отлично}>>. Автору рекомендуется опубликовать новые результаты и продолжить обучение в аспирантуре.

	\textbf{Рецензент}

	\noindent Профессор Отделения машиностроения, морской технологии и транспорта Инженерного департамента Политехнического института ДВФУ

	\vspace{10pt}
	\noindent $\underset{
		\textit{ (должность, уч. звание) }
	}{
		\uline{\text{\hspace{10pt} доцент, к.ф.-м.н. \hspace{10pt}}}
	}$
	\hfill \usersignature{\hspace{10pt} Амосова Е.В. \hspace{10pt}}

	\vspace{10pt}
	\textbf{Дата:} 
	\docdatemonth{\hspace{90pt}}{e}


\end{document}