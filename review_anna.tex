\documentclass[12pt, a4paper, titlepage]{extarticle}
\usepackage{styles/style}
\usepackage{styles/symbols}
\usepackage{styles/titlepage}

\linespread{1.2}
\begin{document}

	\pagestyle{empty}
	
	\drawtitleruler

	\vspace*{-30pt}
	
	\schoolfromold{ИНСТИТУТ МАТЕМАТИКИ И КОМПЬЮТЕРНЫХ ТЕХНОЛОГИЙ}{Департамент математического и компьютерного моделирования}
	
	\vspace*{-10pt}

	\begin{center}

		\textbf{ОТЗЫВ РУКОВОДИТЕЛЯ}

		\vspace*{-4pt}
		на выпускную квалификационную работу студентки

		\vspace{5pt}

		\underlinefill{(Фамилия, Имя, Отчество)}{Шестопаловой Анны Владимировны}

	\end{center}

	\noindent Направление подготовки: \underline{\hside{10pt}{02.03.01 Математика и компьютерные науки}}

	\noindent Профиль: \underline{\hside{10pt}{Сквозные цифровые технологии}}

	\noindent Группа: \underline{\hside{10pt}{Б9120-02.03.01сцт}}

	\noindent Руководитель ВКР: \underlinesimple{
		(ученая степень, ученое звание, Ф.И.О.)
	}{
		\text{\hside{50pt}{ассистент, Бузалов Павел Константинович}}
	}

	\noindent На тему: \underline{\hspace{10pt} <<Разработка компьютерной игры в жанре РПГ с применением нейронных сетей}

	\noindent \underline{и дифференциальных уравнений на движке Unreal Engine>> \hspace{50pt}}

	\begin{flushright}
		Дата защиты ВКР: \docdateall{6}{июля}{24}{e} 
		\hphantom{\hspace{40pt}}
	\end{flushright}

	Цель дипломной работы Шестопаловой А.В. заключается в разработке компьютерной игры в жанре RPG с использованием нейронных сетей и дифференциальных уравнений на движке Unreal Engine 5. В работе проведен анализ концепции игры, проектирование игровых механик и локаций, а также создание систем диалогов и квестов. Внедрены технологии симуляции воды и текстуры ландшафта. Важной особенностью является интеграция нейронной сети Mistral для создания умных NPC, способных вести диалог с игроком в реальном времени.

	В ходе выполнения работы Шестопалова А.В. проявила себя как самостоятельный, дисциплинированный студент, обладающий способностью глубоко вникать в суть задачи и эффективно изучать предметную область. Дипломная работа выполнена в соответствии со всеми требованиями и заслуживает оценки <<отлично>>.

	\vspace{20pt}

	\noindent Оригинальность текста ВКР составляет: \underline{\hside{10pt}{88.84}}\%

	\noindent Оценка: \underline{\hside{10pt}{<<отлично>>}}

	\noindent Руководитель ВКР: \underlinesimple{
		(уч. степень, уч. звание)
	}{
		\text{\hside{10pt}{ассистент}}
	} \signature{\hspace{80pt}} \hspace{30pt} \underlinesimple{(Ф.И.О.)}{\text{\hside{10pt}{Бузалов П.К.}}}

	\vspace{10pt}
	\noindent\docdatemonthsp{\hspace{60pt}}{e}

\end{document}