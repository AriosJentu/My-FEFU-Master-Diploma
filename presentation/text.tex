\documentclass[14pt, a4paper]{extarticle}
\usepackage[russian]{babel}
\usepackage{amsmath}
\usepackage{amssymb}
\usepackage{../symbols}
\usepackage{geometry}

\geometry{left=3cm}
\geometry{right=1.5cm}
\geometry{top=2cm}
\geometry{bottom=2cm}

\begin{document}

	(1) Защищается Максимов Павел Александрович, студент группы М910-01.04.02матмод. Тема выпускной квалификационной работы -- <<Численный анализ диффузионных моделей распространения вирусов>>.

	(2) Начнем с основных положений. В данной работе исследуются краевые и экстремальные задачи для нелинейных стационарных уравнений реакции-диффузии-конвекции, а так же рассматривается численный алгоритм решения начально-краевых задач для систем нелинейных уравнений диффузии-реакции.

	(3) План доклада представлен на слайде.

	(4) Перейдем к постановке краевой задачи. В ограниченной области трехмерного пространства рассматривается следующая краевая задача. Здесь $\varphi$ -- концентрация вещества, $u$ -- заданный вектор скорости, $\lambda$ -- коэффициент диффузии, $f$ -- объемная плотность внешних источников, $k$ -- коэффициент реакции, в общем случае нелинейный.

	(5) Введем следующие функциональные пространства, и положим условия ограниченности области, и принадлежности некоторых функций соответствующим пространствам.

	(6) На коэффициент реакции поставим следующее условие, для которого будет справедливо представленное неравенство.

	\textit{\small (третье условие) Для любой функции $v \in H^1(\Omega)$ справедливо вложение $k(v, \cdot) \in L^{p}_{+}$, для некоторого $p \ge 3/2$, не зависящего от $v$, и на любом шаре $B_r$ выполняется такое неравенство.}

	(7) Так же на коэффициент реакции наложим дополнительно условия на монотонность и ограниченность. 
	%что в некоторой подобласти $k$ -- монотонный и ограниченный в следующем смысле, при этом в другой подобласти справедливо следующее неравенство.

	(8) Перейдем к разрешимости краевой задачи. Рассмотрим пример \textbf{таких} функций в соответствующих областях. Проводя анализ, поставим следующую теорему. \textit{[прочитать]} %При выполнении условий на коэффициент реакции существует слабое решение поставленной краевой задачи, для которого справедлива следующая оценка. Если к тому же выполняется дополнительное условие, то слабое решение будет единственным.

	(9) Теперь рассмотрим похожую задачу, но с другими краевыми условиями. Пусть граница $\Gamma$ состоит из двух частей, на одной из которых ставятся условия Дирихле, на другой - смешанные условия. Здесь $\alpha$ -- коэффицент массообмена, в общем случае нелинейный, и $\chi$ -- объемная плотность граничных источников.

	(10) Вводим следующие функциональные пространства, и такие условия.

	(11) Перейдем к задаче мультипликативного управления. Для постановки задачи управления, разобьем множество исходных данных задачи на две группы: группу управлений, куда отнесем коэффициент диффузии, полагая, что он может изменяться в некоторых множествах, удовлетворяющих следующим условиям, и группу фиксированных данных, куда отнесем все остальные параметры уравнения.
	%группу фиксированных данных, куда отнесем функции скорости, коэффициенты реакции и массообмена, внутренние и внешние источники, и концентрацию на границе, и группу управлений, куда отнесем коэффициент диффузии, полагая, что он может изменяться в некоторых множествах, удовлетворяющих следующим условиям. 
	Поставим задачу управления, и определим следующие функционалы качества. 

	\textit{\small Здесь $\varphi_d$ -- возмущенное состояние концентрации}

	(12) Для конкретных коэффициентов $k$ и $\alpha$, представленных здесь, получены следующие оценки на разность $\lambda$ и $\varphi$ соотвественно.

	(13) Перейдем к разделу численного анализа. В двумерной ограниченной области рассмотрим следующую нелинейную диффузионно-реакционную модель распространения вирусов. Она представляет из себя нестационарную систему нелинейных уравнений со следующими начально-краевыми условиями.

	Здесь $S, I$ и $R$ -- плотностные вероятности восприимчивости к заражению, заражения, и восстановления соответственно, $\kappa$ -- скорость передачи инфекции, $\tau$ -- характеристическое время восстановления, и $D$ -- коэффициент диффузии. Данная система удовлетворяет закону сохранения плотности во времени, а переход между состояниями происходит последовательно от восприимчивых $\to$ к зараженным под действием коэффициента передачи и затем $\to$ к восстановленным под действием коэффициента восстановления.

	(14) Данная модель была предложена в публикации Постникова в 2008м году, и для данной модели был проведен численный анализ на основе симуляций Монте-Карло. В этой же работе, анализ будет проведен на основе метода конечных элементов. Для этого сведем систему к стационарной методом конечных разностей. Получим следующую систему стационарных уравнений.

	(15) Теперь перейдем к параметрам эксперимента. Выберем следующие параметры. В качестве области выберем квадрат размерностью $1 \times 1$ безразмерных единиц. Согласно публикации, положим \textbf{такие} значения, и полагая плотностные вероятности безразмерными, положим следующий закон сохранения. Начальное распределение выберем таким. 

	С физической точки зрения, такие условия удовлетворяют следующей модели: Концентрация основной части зараженных особей в начальный момент времени находится в центре области, и плавно распределена в округе. Все остальные особи в это время являются подверженными. В области в начальный момент времени отсутствуют восстановленные после заболевания особи. На границе происходит следующее: подверженные особи все меньше приходят в зону заражения со временем, и к половине рассматриваемого периода все возможные особи полностью покинут границы области. На границе отсутствуют инфицированные и восстановленные особи, что говорит об изолированности системы относительно зараженных.

	(16) Поставим слабую формулировку для второго уравнения системы, так как первое и третье вырождаются в линейные рекуррентные уравнения относительно текущих состояний.

	(17) Собственно, сам численный эксперимент будет проводить с шагом $h = 0.05$, конечным временем $T = 1$, и размерности сетки $20 \times 20$. Получим следующие результаты.

\end{document}